\documentclass[titlepage, a4paper, 10pt, reqno, openany]{report}
\input{./input/PREAMBLE}
\begin{document}
\renewcommand\thesection{\arabic{section}}
\renewcommand\thesubsection{\thesection.\arabic{subsection}}
\renewcommand\thesubsubsection{\thesection.\thesubsection.\arabic{subsubsection}}
\begin{titlepage}
\title{Posicionamento Painel Fotovoltaico}
\author{
\\
\\
\\
\\
\\
\\
\begin{minipage}{0.4\linewidth}
\flushleft
\textbf{Aluno} : \\
\emph{S\'{e}rgio Santos},\;$N^o$:\; 1020881 \\
\end{minipage}
\hfill
\fbox{
\begin{minipage}{0.5\linewidth}
\centering
\textbf{Docente/Orientador} \\
\begin{flushleft}
Abel António de Azevedo Ferreira, \textit{abe}\\
Lino Manuel Baptista Figueiredo, \textit{lbf}\\
\end{flushleft}
\textbf{Unidade Curricular} \\
LABSIS \\
\end{minipage}
}
}
\date{\rule[180pt]{0pt}{5pt} } %\today
\end{titlepage}
%%%%%%%%%%%%%%%%%%%%%%%%%%%%%%%%%%%%%%%%%%%%%%%%%%%%%%%%%%%%%%%%%%%%
\begin{minipage}{\linewidth}
\includegraphics[scale=0.60]{./image/capa/ISEP_marca_cor_grande.png}
\maketitle
\end{minipage}
%titlepage
%\tableofcontents
%\appendix
\pagestyle{plain}%plain headings empty
%%%%%%%%%%%%%%%%%%%%%%%%%%%%%%%%%%%%%%%%%%%%%%
%few lessons \newline \newpage only in text.
%\pagestyle{headings} %estilo de numeração
%{\tiny text}{\scriptsize text}{\footnotesize text}{\small text}{\normalize text}{\large text}{\Large text}{\huge text}{Huge text}{\it text}{\bf text}{\tt text}\underline{text}
%\newline \newpage \linebreak \\[distance]
%\. \, \; quad qquad hspace{xx cm} vspace{xx cm}
%{\color{cor}text}
%\footcite{•}\twocolumn\onecolumn
% $ formula $ $$ formula $$ \nonumber \\
%\frac{•}{•} \dfrac{•}{•} \sqrt[•]{•} \binom{}{}
%\mathbb{NZqRC} \mathcal{texto} \mathfrak{texto}
%\vert \Vert \leq \geq \neq \to \lim
%displaystyle \int \sum \prod \cup \cap
%\longleftarrow \overline{•} \langle
%\hline \caption \label \centering \ref{Capa}
%%%%%%%%%%%%%%%%%%%%%%%%%%%%%%%%%%%%%%%%%%%%%%%
\chapter*{Led Pisca e circuito Dimmer}
\section{Introdu\c{c}\~{a}o}

Este relat\'{o}rio tem como objetivo a implementa\c{c}\~{a}o de um {\bf Dimmer} controlado por um potenci\'{o}metro [PC0] recorrendo a {\bf PWM} (Power Width Modulation) sendo um {\bf LED} a sua carga [PB1]. \\
O PWM usado \'{e} de oito bits, dai o {\bf ADC} tem que ser adaptado de forma a gerar oito bits para controlar o PWM, neste caso Fast PWM, um LED vai estar a piscar a 1 Hz no PORTD6 permanentemente em simult\^{a}neo. \\
O c\'{o}digo deste trabalho primeiro ser\'{a} feito em Assembler e depois em C. 
%\newpage
\\
\\
Mas primeiro vai ser feito quatro programas apenas concentrado a volta de pôr um led a piscar a {\bf Um} Hertz no PORTD6 em assembly e C por software e depois utilizando interrupções.\\
Isto feito assim para servir como introdução ao microcontrolador da AVR [Atmega88] pelo \textquotedblleft Meu Primeiro Programa \textquotedblright, que consist em por um {\bf LED} (light emitting diode) a piscar no mundo dos {\bf MCU} (Microcontroller Unit). \par
%\newpage

\section{Arquitetura}
O {\bf CPU} (Unidade Central de Processamento) dos microcontroldores da Atmel de 8 e 32 bits s\~{a}o baseados na arquitetura avan\c{c}ada de {\bf Harvard} na qual esta concebido para baixos consumos e performance. \par
Este tipo de arquitetura tem dois  busses (barramentos) um dedicado a leitura das instruções a executar e outra para escrita e leitura de data (informa\c{c}\~{a}o ou dados), isto assegura que uma nova instru\c{c}\~{a}o pode ser executada em cada ciclo de rel\'{o}gio, na qual elimina estados de espera quando n\~{a}o ha instru\c{c}\~{o}es prontas a executar. \par
Nos microcontroladores da AVR os barramentos est\~{a}o configurados de forma a dar prioridade ao barramento das instru\c{c}\~{o}es do CPU acesso a memoria flash enquanto o barramento da CPU de dados tem prioridade de acesso a {\bf SRAM} (Static Random Access Memory). \par
O espa\c{c}o de memoria de dados \'{e} dividida em tr\^{e}s, os {\bf GPR} (General Purpose Registers) as {\bf SFRs} (Special Function Registers) ou memoria de I/O e a data SRAM. \par
\begin{comment}
\begin{figure}[H]
	\centering
	\includegraphics[scale=0.75]{./image/ARCHITECTURE/Harvard_architecture.png}\\
	\caption{Arquitetura Harvard}
	\label{Arquitetura Harvard}
\end{figure}\par
\end{comment}
Os microcontroladores da AVR utiliza uma arquitetura de instru\c{c}\~{o}es {\bf RISC} (Reduced Instruction Set Computer ou Reduced COMPLEXITY Instruction Set Computer) na qual reduz a complexidade dos circuitos na codifica\c{c}\~{a}o de cada instru\c{c}\~{a}o. \par
Dai que os microcontroladores que se baseiam nestes tipos de arquitetura s\~{a}o sinonimo de c\'{o}digo reduzido, alta performance e baixo consumo energ\'{e}tico \par
\begin{comment}
\begin{figure}[H]
	\centering
	\includegraphics[scale=1]{./image/ARCHITECTURE/Block_diagram.png}\\
	\caption{Diagrama de blocos}
	\label{Diagrama de blocos}
\end{figure}
\end{comment}
%\newpage
\section{Hardware}
O micro-controlador a ser usado \'{e} um Atmega88, seu Datasheet (Manual do Componente) \'{e} uma pe\c{c}a fundamental para usar como suporte na sua utiliza\c{c}\~{a}o. \\
\begin{comment}	
\begin{figure}[H]
	\centering
	\includegraphics[scale=0.75]{./image/PACKAGE/Configuracao_pin.png}\\
	\caption{Configura\c{c}\~{a}o dos pinos}
\end{figure}
Para o programar foi ligado uma ficha ISP aos seus respectivos pinos. \\
\begin{figure}[H]
	\centering
	\includegraphics[scale=0.75]{./image/ISP_JTAG/isp_6pin.png}\\
	\caption{Ficha ISP}
\end{figure}\par
\end{comment}
%\newpage
Para Facilitar seu desenvolvimento foi feito o circuito numa placa pre furada, e sua implementa\c{c}\~{a}o podendo ser alimentada por uma fonte DC 12Volt por um Jack.
\begin{figure}[H]
	\centering
	\includegraphics[scale=0.085]{./image/Board/DevBoard_1.JPG}
	%	\caption{Dev Board}
	%	\label{Dev Board}
\end{figure}
Para programa\c{c}\~{a}o e debug do integrado \'{e} usado um Atmel-ICE, uma ferramenta de desenvolvimento que neste caso do i.c Atmega88 tem dispon\'{i}vel programa\c{c}\~{a}o via {\bf ISP} e debug por {\bf debugWIRE}. \\
Os Par\^{a}metros de configura\c{c}\~{a}o do integrado s\~{a}o os seguintes, Device Signature do Atmega88=0x1E930A, {\bf ISP} clock a 125Khz, BOOTZ=1024W\_0C00, SPIEN=ON, BODLEVEL=2V7, SUT\_CKSEL=Int. RC. Osc. 8MHZ\_XX\_16KCK\_14CK\_65MS , EXTENDED FUSE= 0xF9,HIGH FUSE=0xD5, LOW FUSE=0xE2 e Lock bits OFF ou seja 0xFF.\\
%\newpage

\section{Software}
Nesta sec\c{c}\~{a}o \'{e} feito o c\'{o}digo correspondente em por um LED a piscar a 1Hz no PORTD6 e depois o código com o PWM controlado por uma entrada analógica, nos casos mencionados na introdu\c{c}\~{a}o.  \\
A ferramenta de desenvolvimento ({\bf IDE}) utilizado \'{e} o Atmel Studio 7 (vers\~{a}o 7.0.129) \\
Deve-se ter em aten\c{c}\~{a}o que no assembly a rotina JMP e CALL n\~{a}o funcionam no ATmega88 pois n\~{a}o fazem parte do seu conjunto de instru\c{c}\~{o}es, como indicado no seu Datasheet, mas recorrer a RJMP e RCALL respetivamente. \\
{\bf 1Hz}:	
\begin{equation}
\boxed{F_{OCnx}=\frac{F_{clk\_I/O}}{2.N.(1+{OCRnx})}} \quad
\implies \quad {T_{OCRnx}=2.N.(1+OCRNx)\times T_{clk\_I/O}}
\end{equation}
{\bf ADC}:
\[ \boxed{ADC=\frac{V_{IN}.1024}{V_{REF}}} \]
{\bf fast PWM}:
\[ \boxed{F_{OCnxPWM}=\frac{F_{clk\_I/O}}{N.(1+{TOP})}} \quad \textit{ com resolu\c{c}\~{a}o minima} \quad \boxed{R_{FPWM}=\frac{log(TOP+1)}{log(2)}} \] \par
Os parâmetros escolhidos para o ADC s\~{a}o o PINC0 como entrada com clock dividido por 128, o mais lento poss\'{i}vel, a tens\~{a}o de referencia \'{e} VCC com capacidade em Vref. \par
Utilizando fast PWM com N=256 e TOP=255, dai pode-se calcular sua respetiva frequ\^{e}ncia e duty cycle mínimo, pelas formulas.
A saída do PWM \'{e} no pino OC1nA ou PORTB1, e o led a piscar a um Hertz no pino PORTD6, n\~{a}o ha necessidade de o mencionar o c\'{o}digo auto explicativo, mesmo assim.
%\newpage

\subsection{Led Pisca 1Hz em assembly por software.}
\begin{minipage}[T]{.3\linewidth}
.INCLUDE \textquotedblleft M88DEF.INC\textquotedblright \\
.ORG 0 \\
\hspace*{.5cm}	LDI R16, HIGH(RAMEND) \\
\hspace*{.5cm}	OUT SPH, R16 \\
\hspace*{.5cm}	LDI R16, LOW(RAMEND) \\
\hspace*{.5cm}	OUT SPL, R16 \\
\hspace*{.5cm}	LDI R16, (1$<<$6) \\
\hspace*{.5cm}	LDI R17, (1$<<$6) \\
\hspace*{.5cm}	SBI DDRD, 6 \\
BACK: \\
\hspace*{.5cm}	EOR R16, R17 \\
\hspace*{.5cm}	OUT PORTD, R16 \\
\hspace*{.5cm}	RCALL DELAY\_HalfSec \\
\hspace*{.5cm}	RJMP BACK \\
DELAY\_HalfSec: \\
\hspace*{.5cm}	LDI R20, 32 \\
L1:	\hspace{.5cm} LDI R21, 100 \\
L2:	\hspace{.5cm} LDI R22, 250 \\
L3: \hspace{.5cm} \\
\hspace*{.5cm}	NOP \\
\hspace*{.5cm}	NOP \\
\hspace*{.5cm}	DEC R22 \\
\hspace*{.5cm}	BRNE L3 \\
\hspace*{.8cm}	DEC R21 \\
\hspace*{.5cm}	BRNE L2 \\
\hspace*{.8cm}	DEC R20 \\
\hspace*{.5cm}	BRNE L1 \\
\hspace*{.8cm}	RET \\[1ex]% [1ex] adds vertical space
;EOF \\
\end{minipage}
\qquad
\begin{minipage}[c]{.7\linewidth}
\begin{tikzpicture}[node distance = 2cm, auto]
% Define
\tikzstyle{RECTANGLE_1} = [rectangle, rounded corners, minimum width=1cm, minimum height=1cm,text centered, draw=black, fill=green!30]
\tikzstyle{RECTANGLE_2} = [rectangle, draw, align=left, fill=blue!20]
\tikzstyle{RECTANGLE_4} = [rectangle, draw, align=left, fill=orange!60]
\tikzstyle{RECTANGLE_3} = [rectangle, draw, align=left, fill=red!60]
\tikzstyle{DIAMOND_1} = [diamond, draw, align=left, fill=red!80]
\tikzstyle{ARROW} = [thick,->,>=stealth]
\tikzstyle{LINE} = [draw, -latex']
\tikzstyle{MYLINE} = [draw, ->,  thick, shorten <=4pt, shorten >=4pt]
% Place nodes
\node [RECTANGLE_1] (start) {.ORG 0};
\node [RECTANGLE_2, below of=start] (setup) {setup stack pointer\\ inicializar port};
\node [RECTANGLE_4, below of=setup] (TOGGLE) {TOGGLE LED};
\node [RECTANGLE_2, below of=TOGGLE] (R20) {R20=32};
\node [RECTANGLE_2, below of=R20] (R21) {R21=100};
\node [RECTANGLE_2, below of=R21] (R22) {R22=250};
\node [RECTANGLE_3, below of=R22] (DEC22) {DEC R22};
\node [DIAMOND_1, below of=DEC22] (BRNE22) {BRNE};
\node [RECTANGLE_3, left of=BRNE22,xshift=-.5cm] (DEC21) {DEC R21};
\node [DIAMOND_1, below of=DEC21] (BRNE21) {BRNE};
\node [RECTANGLE_3, left of=BRNE21,xshift=-.5cm] (DEC20) {DEC R20};
\node [DIAMOND_1, below of=DEC20] (BRNE20) {BRNE};
% LINES and ARROWS
\path [LINE] (start) -- node (l_1) {} (setup);
\path [LINE] (setup) -- node (l_2) {} (TOGGLE);
\path [LINE] (TOGGLE)-- node (l_3) {} (R20);
\path [LINE] (R20)-- node (l_4) {} (R21);
\path [LINE] (R21)-- node (l_5) {} (R22);
\path [LINE] (R21)-- node (l_5) {} (R22);
\path [LINE] (R22)-- node (l_6) {} (DEC22);
\path [LINE] (DEC22)-- node (l_7) {} (BRNE22);
\path [LINE] (BRNE22)-- node (l_8) {} (DEC21);
\path [LINE] (DEC21)-- node (l_9) {} (BRNE21);
\path [LINE] (BRNE21)-- node (l_10) {} (DEC20);
\path [LINE] (DEC20)-- node (l_11) {} (BRNE20);
\path [LINE] (BRNE20.east)|- +(+7,0) |- (l_4);
\path [LINE] (BRNE21.east)|- +(+4,0) |- (l_5);
\path [LINE] (BRNE22.east)|- +(+1,0) |- (l_6);
\path [LINE] (BRNE20.west)|- +(-.5,0) |- (l_2);
\end{tikzpicture}
\end{minipage}
\\
\\
\\
\\
Cada ciclo de maquina demora $1/8 Mhz$ que \'{e} igual a 125 nano segundos. \\
$Delay=32 \times 100 \times 250 \times 5 \times 125 ns \quad \implies \quad Delay=500ms$ \\
%;Neste calculo n\~{a}o esta inclu\'{i}do o atraso dos cabe\c{c}alhos dos dois ciclos exteriores.
%\newpage

\subsection {Led pisca 1 Hz em C por Software.}	
\begin{minipage}[T]{.3\linewidth}
/***PreProcessor***/ \\
\#ifndef F\_CPU \\
\hspace*{.5cm}	\#define F\_CPU 8000000UL \\
\#endif \\
/***Library***/ \\
\#include \textless avr/io.h \textgreater \\
\#include \textless avr/interrupt.h \textgreater \\
\#include \textless util/delay.h \textgreater \\
\#include \textless avr/pgmspace.h \textgreater \\
/***Define and Macro***/ \\
\#define TRUE 1 \\
/***Global Variable***/ \\
/***Prototype***/ \\
void PORTINIT(void); \\
/***MAIN***/ \\
int main(void) \\
\textbraceleft \\
\hspace*{.5cm}	uint8\_t i,j,k; \\
\hspace*{.5cm}	PORTINIT(); \\
\hspace*{.5cm}    while(TRUE) \\
\hspace*{.5cm}    \textbraceleft \\
\hspace*{1cm}		for(i=32;i;i- -)\textbraceleft \\
\hspace*{1.5cm}			for(j=166;j;j- -)\textbraceleft \\
\hspace*{2cm}				for(k=250;k;k- -); \\
\hspace*{1.5cm}			\textbraceright \\
\hspace*{1cm}		\textbraceright \\
\hspace*{1cm}		PORTD\textasciicircum = (1$<<$PORTD6); \\
\hspace*{.5cm}	\textbraceright \\
\textbraceright \\
/***Procedure and Function***/ \\
void PORTINIT(void)\textbraceleft \\
\hspace*{.5cm}	DDRD=(1$<<$PORTD6); \\
\hspace*{.5cm}	PORTD=(1$<<$PORTD6); \\
\textbraceright \\
/***Interrupt***/ \\
/***EOF***/
\end{minipage}
\qquad
\begin{minipage}[c]{.7\linewidth}
\begin{tikzpicture}[node distance = 2cm, auto]
% Define
\tikzstyle{RECTANGLE_1} = [rectangle, rounded corners, minimum width=1cm, minimum height=1cm,text centered, draw=black, fill=green!30]
\tikzstyle{RECTANGLE_2} = [rectangle, draw, align=left, fill=blue!20]
\tikzstyle{RECTANGLE_11} = [rectangle, draw, align=left, fill=red!10]
\tikzstyle{RECTANGLE_12} = [rectangle, draw, align=left, fill=red!40]
\tikzstyle{RECTANGLE_13} = [rectangle, draw, align=left, fill=red!80]
\tikzstyle{RECTANGLE_3} = [rectangle, draw, align=left, fill=orange!60]
\tikzstyle{DIAMOND_1} = [diamond, draw, align=left, fill=red!10]
\tikzstyle{DIAMOND_2} = [diamond, draw, align=left, fill=red!40]
\tikzstyle{DIAMOND_3} = [diamond, draw, align=left, fill=red!80]
\tikzstyle{ARROW} = [thick,->,>=stealth]
\tikzstyle{LINE} = [draw, -latex']
\tikzstyle{MYLINE} = [draw, ->,  thick, shorten <=4pt, shorten >=4pt]
% Place nodes
\node [RECTANGLE_1] (start) {MAIN};
\node [RECTANGLE_2, below of=start] (setup) {uint8\_t i,j,k; \\ PORTINIT();};
\node [RECTANGLE_11, below of=setup] (for_1_inic) {i=32};
\node [DIAMOND_1, below of=for_1_inic] (for_1) {i==0};
\node [RECTANGLE_12, right of=for_1,xshift=.5cm] (for_2_inic) {j=166};
\node [RECTANGLE_3, left of=for_1,xshift=-1cm] (TOGGLE) {PORTD\textasciicircum \\ = (1$<<$PORTD6);};
\node [DIAMOND_2, below of=for_2_inic] (for_2) {j==0};
\node [RECTANGLE_13, right of=for_2,xshift=.5cm] (for_3_inic) {k=250};
\node [RECTANGLE_11, left of=for_2,xshift=-.5cm] (for_1_update) {i- -};
\node [DIAMOND_3, below of=for_3_inic] (for_3) {k==0};
\node [RECTANGLE_13, right of=for_3,xshift=.5cm] (for_3_update) {k- -};
\node [RECTANGLE_12, left of=for_3,xshift=-.5cm] (for_2_update) {j- -};
% LINES and ARROWS
\path [LINE] (start) -- node (l_1) {} (setup);
\path [LINE] (setup) -- node (l_2) {} (for_1_inic);
\path [LINE] (TOGGLE.west) |- +(-.1,0) |- (l_2);
\path [LINE] (for_1) -- node (l_11) {yes} (TOGGLE);
\path [LINE] (for_1) -- node (l_12) {no} (for_2_inic);
\path [LINE] (for_1_inic) -- node (l_10) {} (for_1);
\path [LINE] (for_1_update.west) |- +(-1,0) |- (l_10);
\path [LINE] (for_2_inic) -- node (l_21) {} (for_2);
\path [LINE] (for_2) -- node (l_20) {no} (for_3_inic);
\path [LINE] (for_2) -- node (l_22) {yes} (for_1_update);
\path [LINE] (for_2_update.west) |- +(-1,0) |- (l_21);
\path [LINE] (for_3_inic) -- node (l_30) {} (for_3);
\path [LINE] (for_3) -- node (l_31) {no} (for_3_update);
\path [LINE] (for_3) -- node (l_33) {yes} (for_2_update);
\path [LINE] (for_3_update) |- node (l_32) {} (l_30);
\end{tikzpicture}
\end{minipage}
\\
\\
\\
Os valores de i, j e k, foram do exerc\'{i}cio anterior que a posterior foi ajustado de forma a obter a frequ\^{e}ncia desejada, com o auxilio de um oscilosc\'{o}pio.
%\newpage

\subsection {Led pisca 1 Hz em Assembly por Interrup\c{c}\~{a}o.}
\begin{minipage}[T]{.3\linewidth}
.EQU REPEAT = 100 \\
.EQU MASK = (1$<<$6) \\
.INCLUDE \textquotedblleft M88DEF.INC\textquotedblright \\[0.5ex]
.ORG 0 \\
\hspace*{.5cm}	RJMP RESET \\
.ORG 0x20 \\
\hspace*{.5cm}	RJMP TIM0\_COMPA \\
.ORG 0x100 \\
RESET: \\
\hspace*{.5cm}	LDI R16, HIGH(RAMEND) \\
\hspace*{.5cm}	OUT SPH, R16 \\
\hspace*{.5cm}	LDI R16, LOW(RAMEND) \\
\hspace*{.5cm}	OUT SPL, R16 \\
\hspace*{.5cm}	LDI R16, MASK \\
\hspace*{.5cm}	LDI R17, MASK \\
\hspace*{.5cm}	LDI R18, REPEAT \\
\hspace*{.5cm}	SBI DDRD, 6 \\
\hspace*{.5cm}	SBI PORTD, 6 \\
\hspace*{.5cm}	RCALL INTRUP\_10ms \\
\hspace*{.5cm}	RJMP MAIN \\
\hspace*{.5cm}	\\
; Setup Periodo 10ms \\
INTRUP\_10ms: \\
\hspace*{.5cm}	LDI R19, (1$<<$OCIE0A) \\
\hspace*{.5cm}	STS TIMSK0, R19 \\
\hspace*{.5cm}	LDI R19, 156 \\
\hspace*{.5cm}	OUT OCR0A, R19 \\
\hspace*{.5cm}	LDI R19, (1$<<$WGM01) \\
\hspace*{.5cm}	OUT TCCR0A, R19 \\
\hspace*{.5cm}	LDI R19, (1$<<$CS02) \\
\hspace*{.5cm}	OUT TCCR0B, R19 \\
\hspace*{.5cm}	SEI \\
\hspace*{.5cm}	\\
MAIN: \\
\hspace*{.5cm}	RJMP MAIN \\
\hspace*{.5cm}	\\
INIC: \\
\hspace*{.5cm}	LDI R18, REPEAT \\
\hspace*{.5cm}	EOR R16, R17 \\
\hspace*{.5cm}	
\hspace*{.5cm}	OUT PORTD, R16 \\
\hspace*{.5cm}	\\
TIM0\_COMPA: \\
\hspace*{.5cm}	DEC R18 \\
\hspace*{.5cm}	BREQ INIC \\
\hspace*{.5cm}	RETI \\
;EOF \\
\end{minipage}
\qquad
\begin{minipage}[c]{.7\linewidth}
\begin{tikzpicture}[node distance = 2cm, auto]
% Define
\tikzstyle{RECTANGLE_1} = [rectangle, align=left, rounded corners, minimum width=1cm, minimum height=1cm,text centered, draw=black, fill=green!30]
\tikzstyle{RECTANGLE_2} = [rectangle, draw, align=left, fill=blue!10]
\tikzstyle{RECTANGLE_3} = [rectangle, draw, align=left, fill=blue!40]
\tikzstyle{RECTANGLE_4} = [rectangle, draw, align=left, fill=purple!60]
\tikzstyle{RECTANGLE_5} = [rectangle, draw, fill=purple!60, text width=5em, text badly centered, minimum height=4em]
\tikzstyle{RECTANGLE_6} = [rectangle, draw, align=left, fill=red!60]
\tikzstyle{RECTANGLE_7} = [rectangle, draw, align=left, fill=orange!60]
\tikzstyle{DIAMOND_1} = [diamond, draw, align=left, fill=red!80]
\tikzstyle{ARROW} = [thick,->,>=stealth]
\tikzstyle{LINE} = [draw, -latex']
\tikzstyle{MYLINE} = [draw, ->,  thick, shorten <=4pt, shorten >=4pt]
% Place node
\node [RECTANGLE_1] (START) {.ORG 0x100};
\node [RECTANGLE_2, below of=START] (SETUP) {Setup Stack Pointer\\ Inicializar variaveis \\ Inicializar port};
\node [RECTANGLE_3, below of=SETUP] (INTERRUPT_SETUP) {Setup Interrupt \\ Parameters};
\node [RECTANGLE_5, below of=INTERRUPT_SETUP] (MAIN) {MAIN};
%
\node [RECTANGLE_1, below= 3cm of MAIN] (INTERRUPT_START) {.ORG 0x20 \\ (Interrupt Sequence)};
\node [RECTANGLE_6, below of=INTERRUPT_START] (DEC18) {DEC R18};
\node [DIAMOND_1, below of=DEC18] (BREQ18) {BREQ};
\node [RECTANGLE_6, right=1cm of BREQ18] (RETI) {RETURN TO MAIN};
\node [RECTANGLE_7, left=1cm of BREQ18] (INIC_COUNTER) {INIC COUNTER \\ (R18=100) \\ TOGGLE};
% LINES and ARROWS
\path [LINE] (START) -- node (l_1) {} (SETUP);
\path [LINE] (SETUP) -- node (l_2) {} (INTERRUPT_SETUP);
\path [LINE] (INTERRUPT_SETUP) -- node (l_3) {} (MAIN);
\path [LINE] (MAIN) |- ($(MAIN.south east) + (0.5,-0.5)$) |- (MAIN);
%
\path [LINE] (INTERRUPT_START) -- node (l_4) {} (DEC18);
\path [LINE] (DEC18) -- node (l_5) {} (BREQ18);
\path [LINE] (BREQ18) -- node (l_6) {no} (RETI);
\path [LINE] (BREQ18) -- node (l_7) {yes} (INIC_COUNTER);
\path [LINE] (INIC_COUNTER) |- ($(INIC_COUNTER.south east) + (5,-1)$) -- (RETI.south);
\end{tikzpicture}
\end{minipage} \\
\\
\\
\\
Usando Par\^{a}metros TIMSK0 com OCIE0A activado, wavegenmode=CTC, OCR0A=156 e por ultimo N=256, para obter um Per\'{i}odo de 10ms com oscila\c{c}\~{a}o por cristal $F_{clk\_I/O}=8Mhz$.

%\newpage

\subsection {Led pisca 1 Hz em C por Interrup\c{c}\~{a}o.}
\begin{minipage}[T]{.3\linewidth}
/***PreProcessor***/ \\
\#ifndef F\_CPU \\
\hspace*{1cm} \#define F\_CPU 8000000UL \\
\#endif \\
/***Library***/ \\
\#include \textless avr/interrupt.h \textgreater \\
\#include \textless util/delay.h \textgreater \\
\#include \textless avr/io.h \textgreater \\
\#include \textless avr/pgmspace.h \textgreater \\
/***Define and macro***/ \\
\#define TRUE 1 \\
\#define REPEAT 100 \\
/***Global variable***/ \\
int count; \\
/***Prototype***/ \\
void PORTINIT(void); \\
void TIMER0ASETUP(void); \\
/***MAIN***/ \\
int main(void) \\
\textbraceleft \\
\hspace*{.5cm}	PORTINIT(); \\
\hspace*{.5cm}	TIMER0ASETUP(); \\
\hspace*{.5cm}	count = REPEAT; \\
\hspace*{.5cm}	while(TRUE) \\
\hspace*{.5cm}	\textbraceleft \\
\hspace*{1cm}   \\
\hspace*{.5cm}	\textbraceright \\
\textbraceright \\
/***Procedure and function***/ \\
void PORTINIT(void)\textbraceleft \\
\hspace*{.5cm}	DDRD = (1$<<$PORTD6); \\
\hspace*{.5cm}	PORTD = (1$<<$PORTD6); \\
\textbraceright \\
void TIMER0ASETUP(void)\textbraceleft \\
\hspace*{.5cm}	uint8\_{t} sreg; \\
\hspace*{.5cm}	sreg = SREG; \\
\hspace*{.5cm}	cli(); \\
/***Periodo de 10ms***/ \\
\hspace*{.5cm}	TCCR0A = (1$<<$WGM01); \\
\hspace*{.5cm}	TIMSK0 = (1$<<$OCIE0A); \\
\hspace*{.5cm}	OCR0A = 156; \\
\hspace*{.5cm}	TCCR0B \textbar = (1$<<$CS02); \\
\hspace*{.5cm}	SREG = sreg; \\
\hspace*{.5cm}	sei(); \\
\textbraceright \\
/***Interrupt***/ \\
ISR(TIMER0{\_}COMPA{\_}vect)\textbraceleft \\
\hspace*{.5cm}	count- -; \\
\hspace*{.5cm}	if(!count)\textbraceleft \\
\hspace*{1cm}		PORTD\textasciicircum = (1$<<$PORTD6); \\
\hspace*{1cm}		count = REPEAT; \\
\hspace*{.5cm}	\textbraceright \\
\textbraceright \\
/***EOF***/
\end{minipage}
\qquad
\begin{minipage}[c]{.7\linewidth}
\begin{tikzpicture}[node distance = 2cm, auto]
% Define
\tikzstyle{RECTANGLE_1} = [rectangle, align=left, rounded corners, minimum width=1cm, minimum height=1cm,text centered, draw=black, fill=green!30]
\tikzstyle{RECTANGLE_2} = [rectangle, draw, align=left, fill=blue!10]
\tikzstyle{RECTANGLE_3} = [rectangle, draw, align=left, fill=blue!40]
\tikzstyle{RECTANGLE_4} = [rectangle, draw, align=left, fill=purple!60]
\tikzstyle{RECTANGLE_5} = [rectangle, draw, fill=purple!60, text width=5em, text badly centered, minimum height=4em]
\tikzstyle{RECTANGLE_6} = [rectangle, draw, align=left, fill=red!60]
\tikzstyle{RECTANGLE_7} = [rectangle, draw, align=left, fill=orange!60]
\tikzstyle{DIAMOND_1} = [diamond, draw, align=left, fill=red!80]
\tikzstyle{ARROW} = [thick,->,>=stealth]
\tikzstyle{LINE} = [draw, -latex']
\tikzstyle{MYLINE} = [draw, ->,  thick, shorten <=4pt, shorten >=4pt]
% Place node
\node [RECTANGLE_1] (START) {MAIN};
\node [RECTANGLE_2, below of=START] (SETUP) {PORTINIT();\\ TIMER0ASETUP();\\ count=REPEAT;};
\node [RECTANGLE_5, below of=SETUP] (LOOP) {While Loop};%

\node [RECTANGLE_1, below= 3cm of LOOP] (INTERRUPT_START) {TIMER INTERRUPT ROUTINE};
\node [RECTANGLE_6, below of=INTERRUPT_START] (COUNT) {count- -};
\node [DIAMOND_1, below of=COUNT] (COUNTNULL) {!count};
\node [RECTANGLE_6, right=1cm of COUNTNULL] (RETURN) {RETURN};
\node [RECTANGLE_7, left=1cm of COUNTNULL] (INIC_COUNTER) {PORTD\^{}=(1$<<$PORTD6) \\ count=REPEAT;};
% LINES and ARROWS
\path [LINE] (START) -- node (l_1) {} (SETUP);
\path [LINE] (SETUP) -- node (l_2) {} (LOOP);
\path [LINE] (LOOP) |- ($(LOOP.south east) + (0.5,-0.5)$) |- (LOOP);
%
\path [LINE] (INTERRUPT_START) -- node (l_4) {} (COUNT);
\path [LINE] (COUNT) -- node (l_5) {} (COUNTNULL);
\path [LINE] (COUNTNULL) -- node (l_6) {no} (RETURN);
\path [LINE] (COUNTNULL) -- node (l_7) {yes} (INIC_COUNTER);
\path [LINE] (INIC_COUNTER) |- ($(INIC_COUNTER.south east) + (4,-1)$) -- (RETURN.south);
\end{tikzpicture}
\end{minipage} \\
\\
\\
$100 \times$ Periodo de 10ms $\implies$ 1sec de Periodo ou 1Hz
%\& \% \$ \# \_ \{ \} \textbackslash \textasciitilde \textasciicircum \textbackslash \par
%\newpage
%%%%%%%%%%%%%%%%%%%%%%%%%%%%%%%%%2ª Parte%%%%%%%%%%%%%%%%%%%%%%%%%%%%%%%%%%%%%%%%%%%%%%%%

\subsection{C\'{o}digo DIMMER e led Pisca em Assembler}
\begin{minipage}[t]{.45\linewidth}
	\scriptsize
	;assembly code \\
	.EQU REPEAT = 100 \\
	.EQU MASK = (1$<<$PD6) \\
	.INCLUDE \textquotedblleft M88DEF.INC\textquotedblright \\
	.ORG 0 \\
	\hspace*{.5cm}	RJMP RESET \\
	.ORG 0x015 \\
	\hspace*{.5cm}	RJMP ADC\_vect \\
	.ORG 0X00E \\
	\hspace*{.5cm}	RJMP TIM0\_COMPA\_vect \\
	.ORG 0x100 \\
	RESET: \\
	\hspace*{.5cm}	LDI R16, HIGH(RAMEND) \\
	\hspace*{.5cm}	OUT SPH, R16 \\
	\hspace*{.5cm}	LDI R16, LOW(RAMEND) \\
	\hspace*{.5cm}	OUT SPL, R16 \\
	\hspace*{.5cm}	LDI R16, MASK \\
	\hspace*{.5cm}	LDI R17, MASK \\
	\hspace*{.5cm}	LDI R18, REPEAT \\
	\hspace*{.5cm}	SBI DDRD, 6 \\
	\hspace*{.5cm}	CBI DDRB, (1$<<$PB1) \\
	\hspace*{.5cm}	CBI PORTD, 6 \\
	\hspace*{.5cm}	RCALL TIMER0A\_setup \\
	\hspace*{.5cm}	RCALL TIMER1A\_setup \\
	\hspace*{.5cm}	RCALL ADC\_setup \\
	\hspace*{.5cm}	RJMP MAIN \\
	\newline
	;--- 10ms Setup \\
	TIMER0A\_setup: \\
	\hspace*{.5cm}	LDI R19, (1$<<$OCIE0A) \\
	\hspace*{.5cm}	STS TIMSK0, R19 \\
	\hspace*{.5cm}	LDI R19, 156 ;OCR0A \\
	\hspace*{.5cm}	OUT OCR0A, R19 \\
	\hspace*{.5cm}	LDI R19, (1$<<$WGM01) \\
	\hspace*{.5cm}	OUT TCCR0A, R19 \\
	\hspace*{.5cm}	LDI R19, (1$<<$CS02) \\
	\hspace*{.5cm}	OUT TCCR0B, R19 \\
	\hspace*{.5cm}	RET
	\newline
	TIMER1A\_setup: \\
	\hspace*{.5cm}	LDI R19, (1$<<$WGM10) \\
	\hspace*{.5cm}	ORI R19, (3$<<$COM1A0) \\
	\hspace*{.5cm}	STS TCCR1A, R19 \\
	\hspace*{.5cm}	LDI R19, (1$<<$WGM12) \\
	\hspace*{.5cm}	STS TCCR1B, R19 \\
	\hspace*{.5cm}	SBI DDRB, PB1 \\
	\hspace*{.5cm}	LDI R19, 128 \\
	\hspace*{.5cm}	STS OCR1AL, R19 \\
	\hspace*{.5cm}	LDS R19, TCCR1B \\
	\hspace*{.5cm}	ORI R19, (1$<<$CS12) \\
	\hspace*{.5cm}	STS TCCR1B, R19 \\
	\hspace*{.5cm}	RET \\
	\newline
	ADC\_setup: \\
	\hspace*{.5cm}	CBI DDRC, PC0 \\
	\hspace*{.5cm}	LDI R19, (1$<<$REFS0) \\
	\hspace*{.5cm}	ORI R19, (1$<<$ADLAR) \\
	\hspace*{.5cm}	STS ADMUX, R19 \\
	\hspace*{.5cm}	LDI R19, (7$<<$ADPS0) \\
	\hspace*{.5cm}	ORI R19, (1$<<$ADIE) \\
	\hspace*{.5cm}	ORI R19, (1$<<$ADEN) \\
	\hspace*{.5cm}	ORI R19, (1$<<$ADSC) \\
	\hspace*{.5cm}	ORI R19, (1$<<$ADATE) \\
	\hspace*{.5cm}	STS ADCSRA, R19 \\
	\hspace*{.5cm}	SEI \\
	\hspace*{.5cm}	RET
\end{minipage}
\vline \quad
\begin{minipage}[t]{.45\linewidth}
	\scriptsize
	MAIN: \\
	\hspace*{.5cm}	RJMP MAIN \\
	\newline
	.ORG 400 \\
	TIM0\_COMPA\_vect: \\
	\hspace*{.5cm}	DEC R18 \\
	\hspace*{.5cm}	BREQ INIC \\
	\hspace*{.5cm}	RETI \\
	INIC: \\
	\hspace*{.5cm}	LDI R18, REPEAT \\
	\hspace*{.5cm}	EOR R16, R17 \\
	\hspace*{.5cm}	OUT PORTD, R16 \\
	\newline
	.ORG 500 \\
	ADC\_vect: \\
	\hspace*{.5cm}	LDS R20, ADCL \\
	\hspace*{.5cm}	LDS R20, ADCH \\
	\hspace*{.5cm}	STS OCR1AL, R20 \\
	\hspace*{.5cm}	RETI \\
	;EOF
\end{minipage} \par
%%%%%%%%%%%%%%%%%%%%%%%%%%%%%%%%%%%%%%%%%%%%%%%%%%%%%%%%%%%%%%%%%%%%%%%%%%%%%%%%%%%%%%%%%%%%%%%%%%%%%%%%%%%%%%%%%%%%%%
\begin{minipage}[c]{.45\linewidth}
	\normalsize
	\begin{tikzpicture}[node distance = 2cm, auto]
		% Define
		\tikzstyle{RECTANGLE_1} = [rectangle, align=left, rounded corners, minimum width=1cm, minimum height=1cm,text centered, draw=black, fill=green!30]
		\tikzstyle{RECTANGLE_2} = [rectangle, draw, align=left, fill=blue!10]
		\tikzstyle{RECTANGLE_3} = [rectangle, draw, align=left, fill=blue!40]
		\tikzstyle{RECTANGLE_4} = [rectangle, draw, align=left, fill=purple!60]
		\tikzstyle{RECTANGLE_5} = [rectangle, draw, fill=purple!60, text width=5em, text badly centered, minimum height=4em]
		\tikzstyle{RECTANGLE_6} = [rectangle, draw, align=left, fill=red!60]
		\tikzstyle{RECTANGLE_7} = [rectangle, draw, align=left, fill=orange!60]
		\tikzstyle{DIAMOND_1} = [diamond, draw, align=left, fill=red!80]
		\tikzstyle{ARROW} = [thick,->,>=stealth]
		\tikzstyle{LINE} = [draw, -latex']
		\tikzstyle{MYLINE} = [draw, ->,  thick, shorten <=4pt, shorten >=4pt]
		% Place node
		\node [RECTANGLE_1] (START) {.ORG 0x100};
		\node [RECTANGLE_2, below of=START] (SETUP) {Setup Stack Pointer\\ Inicializar variaveis \\ Inicializar port};
		\node [RECTANGLE_3, below of=SETUP] (INTERRUPT_SETUP) {TIMER0A\_setup \\ TIMER1A\_setup \\ ADC\_setup};
		\node [RECTANGLE_5, below of=INTERRUPT_SETUP] (MAIN) {MAIN};
		% LINES and ARROWS
		\path [LINE] (START) -- node (l_1) {} (SETUP);
		\path [LINE] (SETUP) -- node (l_2) {} (INTERRUPT_SETUP);
		\path [LINE] (INTERRUPT_SETUP) -- node (l_3) {} (MAIN);
		\path [LINE] (MAIN) |- ($(MAIN.south east) + (0.5,-0.5)$) |- (MAIN);
	\end{tikzpicture}
\end{minipage}
\quad
\begin{minipage}[c]{.45\linewidth}
	\normalsize
	\begin{tikzpicture}[node distance = 2cm, auto]
		% Define
		\tikzstyle{RECTANGLE_1} = [rectangle, align=left, rounded corners, minimum width=1cm, minimum height=1cm,text centered, draw=black, fill=green!30]
		\tikzstyle{RECTANGLE_2} = [rectangle, draw, align=left, fill=blue!10]
		\tikzstyle{RECTANGLE_3} = [rectangle, draw, align=left, fill=blue!40]
		\tikzstyle{RECTANGLE_4} = [rectangle, draw, align=left, fill=purple!60]
		\tikzstyle{RECTANGLE_5} = [rectangle, draw, fill=purple!60, text width=5em, text badly centered, minimum height=4em]
		\tikzstyle{RECTANGLE_6} = [rectangle, draw, align=left, fill=red!60]
		\tikzstyle{RECTANGLE_7} = [rectangle, draw, align=left, fill=orange!60]
		\tikzstyle{DIAMOND_1} = [diamond, draw, align=left, fill=red!80]
		\tikzstyle{ARROW} = [thick,->,>=stealth]
		\tikzstyle{LINE} = [draw, -latex']
		\tikzstyle{MYLINE} = [draw, ->,  thick, shorten <=4pt, shorten >=4pt]
		% Place node
		\node [RECTANGLE_1, below= 3cm of MAIN] (INTERRUPT_START) {.ORG 400};
		\node [RECTANGLE_6, below of=INTERRUPT_START] (DEC18) {DEC R18};
		\node [DIAMOND_1, below of=DEC18] (BREQ18) {BREQ};
		\node [RECTANGLE_6, right=1cm of BREQ18] (RETI) {RETI};
		\node [RECTANGLE_7, left=1cm of BREQ18] (INIC_COUNTER) {INIC COUNTER \\ (R18=100) \\ TOGGLE};
		% LINES and ARROWS
		\path [LINE] (INTERRUPT_START) -- node (l_4) {} (DEC18);
		\path [LINE] (DEC18) -- node (l_5) {} (BREQ18);
		\path [LINE] (BREQ18) -- node (l_6) {no} (RETI);
		\path [LINE] (BREQ18) -- node (l_7) {yes} (INIC_COUNTER);
		\path [LINE] (INIC_COUNTER) |- ($(INIC_COUNTER.south east) + (4,-1)$) -- (RETI.south);
	\end{tikzpicture}
	\newline \newline
	\begin{tikzpicture}[node distance = 2cm, auto]
		% Define
		\tikzstyle{RECTANGLE_1} = [rectangle, align=left, rounded corners, minimum width=1cm, minimum height=1cm,text centered, draw=black, fill=green!30]
		\tikzstyle{RECTANGLE_2} = [rectangle, draw, align=left, fill=blue!10]
		\tikzstyle{RECTANGLE_3} = [rectangle, draw, align=left, fill=blue!40]
		\tikzstyle{RECTANGLE_4} = [rectangle, draw, align=left, fill=purple!60]
		\tikzstyle{RECTANGLE_5} = [rectangle, draw, fill=purple!60, text width=5em, text badly centered, minimum height=4em]
		\tikzstyle{RECTANGLE_6} = [rectangle, draw, align=left, fill=red!60]
		\tikzstyle{RECTANGLE_7} = [rectangle, draw, align=left, fill=orange!60]
		\tikzstyle{DIAMOND_1} = [diamond, draw, align=left, fill=red!80]
		\tikzstyle{ARROW} = [thick,->,>=stealth]
		\tikzstyle{LINE} = [draw, -latex']
		\tikzstyle{MYLINE} = [draw, ->,  thick, shorten <=4pt, shorten >=4pt]
		% Place node
		\node [RECTANGLE_1, below= 3cm of MAIN] (INTERRUPT_START) {.ORG 500};
		\node [RECTANGLE_6, below of=INTERRUPT_START] (DEC18) {OCR1AL=ADCH};
		\node [RECTANGLE_6, below of=DEC18] (RETI) {RETI};
		% LINES and ARROWS
		\path [LINE] (INTERRUPT_START) -- node (l_4) {} (DEC18);
		\path [LINE] (DEC18) -- node (l_5) {} (RETI);
	\end{tikzpicture}
\end{minipage}
%\newpage

\subsection {C\'{o}digo DIMMER e led Pisca em C}
\begin{minipage}[t]{.35\linewidth} 
	\tiny
	/***Pre Processor***/ \\
	\#ifndef F\_CPU \\
	\hspace*{.5cm}	\#define F\_CPU 8000000UL \\
	\#endif \\
	/***Library***/ \\
	\#include \textless avr/io.h \textgreater \\
	\#include \textless avr/interrupt.h \textgreater \\
	\#include \textless util/delay.h \textgreater \\
	\#include \textless avr/pgmspace.h \textgreater \\
	/***Define and Macro***/ \\
	\#define TRUE 1 \\
	\#define REPEAT 100 \\
	/***Global variable***/ \\
	static volatile uint16\_t adcL\_tmp; \\
	static volatile uint16\_t adcH\_tmp; \\
	int count; \\
	/***Prototype***/ \\
	void PORTINIT(void); \\
	/******/ \\
	void TIMER0A\_SETUP(void); \\
	/******/ \\
	void TIMER1A\_SETUP(uint8\_t wagenmode, uint8\_t compoutAmode,uint8\_t interruptmask); \\
	void TIMER1A\_STARTSTOP(uint16\_t prescaler); \\
	void TIMER1A\_trigger(uint16\_t Atrigger); \\
	void TIMER1\_intcapture(uint16\_t capture); \\
	/******/ \\
	void ADC\_ENABLE(uint8\_t ADC\_channel, uint8\_t ADC\_Vref, uint8\_t ADC\_lar, uint8\_t ADC\_Div); \\
	/***MAIN\_MAIN\_MAIN***/ \\
	int main(void) \\
	\textbraceleft \\
	\hspace*{.5cm}	//PORTINIT(); \\
	\hspace*{.5cm}	/******/ \\
	\hspace*{.5cm}	TIMER0A\_SETUP(); \\
	\hspace*{.5cm}	count=REPEAT; \\
	\hspace*{.5cm}	/******/ \\
	\hspace*{.5cm}	TIMER1A\_SETUP(5,2,0); \\
	\hspace*{.5cm}	TIMER1A\_trigger(128); \\
	\hspace*{.5cm}	TIMER1A\_STARTSTOP(256); \\
	\hspace*{.5cm}	/******/ \\
	\hspace*{.5cm}	ADC\_ENABLE(0, 1, 1, 128); \\
	\hspace*{.5cm}    while(TRUE) \\
	\hspace*{.5cm}    \textbraceleft \\
	\hspace*{.5cm}    \textbraceright \\
	\textbraceright \\
	/***Procedure and Function***/ \\
	void PORTINIT(void) \\
	\textbraceleft \\
	\textbraceright \\
	/*****/ \\
	void TIMER0A\_SETUP(void) \\
	\textbraceleft \\
	\hspace*{.5cm}	DDRD=(1$<<$PORTD6); \\
	\hspace*{.5cm}	/***T=10ms***/ \\
	\hspace*{.5cm}	TCCR0A=(1$<<$WGM01); \\
	\hspace*{.5cm}	TIMSK0=(1$<<$OCIE0A); \\
	\hspace*{.5cm}	OCR0A=156; \\
	\hspace*{.5cm}	TCCR0B\textbar =(1$<<$CS02); \\
	\textbraceright \\
	/*****/ \\
	void TIMER1A\_SETUP(uint8\_t wavegenmode, uint8\_t compoutAmode,uint8\_t interruptmask) \\
	\textbraceleft \\
	\hspace*{.5cm}	uint8\_t TCCR1A\_tmp=0x00; \\
	\hspace*{.5cm}	uint8\_t TCCR1B\_tmp=0x00; \\
	\hspace*{.5cm}	uint8\_t TIMSK1\_tmp=0x00; \\
	\hspace*{.5cm}	switch(wavegenmode)\textbraceleft \\
	\hspace*{1cm}		case 0: \\
	\hspace*{1.5cm}			break; \\
	\hspace*{1cm}		case 1: \\
	\hspace*{1.5cm}			TCCR1A\_tmp=(1$<<$WGM10); \\
	\hspace*{1.5cm}			break; \\
	\hspace*{1cm}		case 2: \\
	\hspace*{1.5cm}			TCCR1A\_tmp=(1$<<$WGM11); \\
	\hspace*{1.5cm}			break; \\
	\hspace*{1cm}		case 3: \\
	\hspace*{1.5cm}			TCCR1A\_tmp=(3$<<$WGM10); \\
	\hspace*{1.5cm}			break; \\
	\hspace*{1cm}		case 4: \\
	\hspace*{1.5cm}			TCCR1B\_tmp=(1$<<$WGM12); \\
	\hspace*{1.5cm}			break; \\
	\hspace*{1cm}		case 5: \\
	\hspace*{1.5cm}			TCCR1A\_tmp=(1$<<$WGM10); \\
	\hspace*{1.5cm}			TCCR1B\_tmp=(1$<<$WGM12); \\
	\hspace*{1.5cm}			break; \\
	\hspace*{1cm}		case 6: \\
	\hspace*{1.5cm}			TCCR1A\_tmp=(1$<<$WGM11); \\
	\hspace*{1.5cm}			TCCR1B\_tmp=(1$<<$WGM12); \\
	\hspace*{1.5cm}			break; \\
	\hspace*{1cm}		case 7: \\
	\hspace*{1.5cm}			TCCR1A\_tmp=(3$<<$WGM10); \\
	\hspace*{1.5cm}			TCCR1B\_tmp=(1$<<$WGM12); \\
	\hspace*{1.5cm}			break; \\
	\hspace*{1cm}		case 8: \\
	\hspace*{1.5cm}			TCCR1B\_tmp=(1$<<$WGM13); \\
	\hspace*{1.5cm}			break; \\
	\hspace*{1cm}		case 9: \\
	\hspace*{1.5cm}			TCCR1A\_tmp=(1$<<$WGM10); \\
	\hspace*{1.5cm}			TCCR1B\_tmp=(1$<<$WGM13); \\
	\hspace*{1.5cm}			break; \\
	\hspace*{1cm}		case 10: \\
	\hspace*{1.5cm}			TCCR1A\_tmp=(1$<<$WGM11); \\
	\hspace*{1.5cm}			TCCR1B\_tmp=(1$<<$WGM13); \\
	\hspace*{1.5cm}			break; \\
	\hspace*{1cm}		case 11: \\
	\hspace*{1.5cm}			TCCR1A\_tmp=(3$<<$WGM10); \\
	\hspace*{1.5cm}			TCCR1B\_tmp=(1$<<$WGM13); \\
	\hspace*{1.5cm}			break; \\
	\hspace*{1cm}		case 12: \\
	\hspace*{1.5cm}			TCCR1B\_tmp=(3$<<$WGM12); \\
	\hspace*{1.5cm}			break; \\
	\hspace*{1cm}		case 14: \\
	\hspace*{1.5cm}			TCCR1A\_tmp=(1$<<$WGM11); \\
	\hspace*{1.5cm}			TCCR1B\_tmp=(3$<<$WGM12); \\
	\hspace*{1.5cm}			break; \\
	\hspace*{1cm}		case 15: \\
	\hspace*{1.5cm}			TCCR1A\_tmp=(3$<<$WGM10); \\
	\hspace*{1.5cm}			TCCR1B\_tmp=(3$<<$WGM12); \\
	\hspace*{1.5cm}			break;
\end{minipage}
\vline \quad
\begin{minipage}[t]{.35\linewidth}
	\tiny
	\hspace*{1cm}		default: \\
	\hspace*{1.5cm}			break; \\
	\hspace*{.5cm}	\textbraceright ; \\
	\hspace*{.5cm}	switch(compoutAmode)\textbraceleft \\
	\hspace*{1cm}		case 0: \\
	\hspace*{1.5cm}			break; \\
	\hspace*{1cm}		case 1: \\
	\hspace*{1.5cm}			TCCR1A\_tmp \textbar =(1$<<$COM1A0); \\
	\hspace*{1.5cm}			break; \\
	\hspace*{1cm}		case 2: \\
	\hspace*{1.5cm}			TCCR1A\_tmp \textbar =(1$<<$COM1A1); \\
	\hspace*{1.5cm}			break; \\
	\hspace*{1cm}		case 3: \\
	\hspace*{1.5cm}			TCCR1A\_tmp \textbar =(3$<<$COM1A0); \\
	\hspace*{1.5cm}			break; \\
	\hspace*{1cm}		default: \\
	\hspace*{1.5cm}			break; \\
	\hspace*{.5cm}	\textbraceright ; \\
	\hspace*{.5cm}	switch(interruptmask)\textbraceleft \\
	\hspace*{1cm}		case 0: \\
	\hspace*{1.5cm}			break; \\
	\hspace*{1cm}		case 1: \\
	\hspace*{1.5cm}			TIMSK1\_tmp=(1$<<$TOIE1); \\
	\hspace*{1.5cm}			break; \\
	\hspace*{1cm}		case 2: \\
	\hspace*{1.5cm}			TIMSK1\_tmp=(1$<<$OCIE1A); \\
	\hspace*{1.5cm}			break; \\
	\hspace*{1cm}		case 3: \\
	\hspace*{1.5cm}			TIMSK1\_tmp=(3$<<$TOIE1); \\
	\hspace*{1.5cm}			break; \\
	\hspace*{1cm}		case 4: \\
	\hspace*{1.5cm}			TIMSK1\_tmp=(1$<<$OCIE1B); \\
	\hspace*{1.5cm}			break; \\
	\hspace*{1cm}		case 5: \\
	\hspace*{1.5cm}			TIMSK1\_tmp=((1$<<$OCIE1B) \textbar (1$<<$TOIE1)); \\
	\hspace*{1.5cm}			break; \\
	\hspace*{1cm}		case 6: \\
	\hspace*{1.5cm}			TIMSK1\_tmp=((3$<<$OCIE1A)); \\
	\hspace*{1.5cm}			break; \\
	\hspace*{1cm}		case 7: \\
	\hspace*{1.5cm}			TIMSK1\_tmp=(7$<<$TOIE1); \\
	\hspace*{1.5cm}			break; \\
	\hspace*{1cm}		case 8: \\
	\hspace*{1.5cm}			TIMSK1\_tmp=(1$<<$ICIE1); \\
	\hspace*{1.5cm}			break; \\
	\hspace*{1cm}		case 9: \\
	\hspace*{1.5cm}			TIMSK1\_tmp=((1$<<$ICIE1) \textbar (1$<<$TOIE1)); \\
	\hspace*{1.5cm}			break; \\
	\hspace*{1cm}		case 10: \\
	\hspace*{1.5cm}			TIMSK1\_tmp=((1$<<$ICIE1) \textbar (1$<<$OCIE1A)); \\
	\hspace*{1.5cm}			break; \\
	\hspace*{1cm}		case 11: \\
	\hspace*{1.5cm}			TIMSK1\_tmp=((1$<<$ICIE1) \textbar (3$<<$TOIE1)); \\
	\hspace*{1.5cm}			break; \\
	\hspace*{1cm}		case 12: \\
	\hspace*{1.5cm}			TIMSK1\_tmp=((1$<<$ICIE1) \textbar (1$<<$OCIE1B)); \\
	\hspace*{1.5cm}			break; \\
	\hspace*{1cm}		case 13: \\
	\hspace*{1.5cm}			TIMSK1\_tmp=((1$<<$ICIE1) \textbar (1$<<$OCIE1B) \textbar (1$<<$TOIE1)); \\
	\hspace*{1.5cm}			break; \\
	\hspace*{1cm}		case 14: \\
	\hspace*{1.5cm}			TIMSK1\_tmp=((1$<<$ICIE1) \textbar (3$<<$OCIE1A)); \\
	\hspace*{1.5cm}			break; \\
	\hspace*{1cm}		case 15: \\
	\hspace*{1.5cm}			TIMSK1\_tmp=((1$<<$ICIE1) \textbar (7$<<$TOIE1)); \\
	\hspace*{1.5cm}			break; \\
	\hspace*{1cm}		default: \\
	\hspace*{1.5cm}			break; \\
	\hspace*{.5cm}	\textbraceright ; \\
	\hspace*{.5cm}	TCCR1A=TCCR1A\_tmp; \\
	\hspace*{.5cm}	TCCR1B=TCCR1B\_tmp; \\
	\hspace*{.5cm}	if(compoutAmode)\textbraceleft \\
	\hspace*{1cm}		DDRB \textbar =(1$<<$PB1); \\
	\hspace*{.5cm}	\textbraceright \\
	\hspace*{.5cm}	TIMSK1=TIMSK1\_tmp; \\
	\textbraceright \\
	void TIMER1A\_STARTSTOP(uint16\_t prescaler) \\
	\textbraceleft \\
	\hspace*{.5cm}	uint8\_t TCCR1B\_tmp=0x00; \\
	\hspace*{.5cm}	switch(prescaler)\textbraceleft \\
	\hspace*{1cm}		case 0: \\
	\hspace*{1.5cm}			TCCR1B\&=\texttildelow ((1$<<$CS12) \textbar (1$<<$CS11) \textbar (1$<<$CS10)); \\
	\hspace*{1.5cm}			break; \\
	\hspace*{1cm}		case 1: \\
	\hspace*{1.5cm}			TCCR1B\_tmp=(1$<<$CS10); \\
	\hspace*{1.5cm}			break; \\
	\hspace*{1cm}		case 8: \\
	\hspace*{1.5cm}			TCCR1B\_tmp=(1$<<$CS11); \\
	\hspace*{1.5cm}			break; \\
	\hspace*{1cm}		case 64: \\
	\hspace*{1.5cm}			TCCR1B\_tmp=(3$<<$CS10); \\
	\hspace*{1.5cm}			break; \\
	\hspace*{1cm}		case 256: \\
	\hspace*{1.5cm}			TCCR1B\_tmp=(1$<<$CS12); \\
	\hspace*{1.5cm}			break; \\
	\hspace*{1cm}		case 1024: \\
	\hspace*{1.5cm}			TCCR1B\_tmp=((1$<<$CS12) \textbar (1$<<$CS10)); \\
	\hspace*{1.5cm}			break; \\
	\hspace*{1cm}		case 2: \\
	\hspace*{1.5cm}			TCCR1B\_tmp=(3$<<$CS11); \\
	\hspace*{1.5cm}			break; \\
	\hspace*{1cm}		case 3: \\
	\hspace*{1.5cm}			TCCR1B\_tmp=((1$<<$CS12) \textbar (1$<<$CS10)); \\
	\hspace*{1.5cm}			break; \\
	\hspace*{1cm}		default: \\
	\hspace*{1.5cm}			TCCR1B\_tmp=(7$<<$CS10); \\
	\hspace*{.5cm}	\textbraceright ; \\
	\hspace*{.5cm}	TCCR1B|=TCCR1B\_tmp; \\
	\textbraceright \\
	void TIMER1A\_trigger(uint16\_t Atrigger) \\
	\textbraceleft \\
	\hspace*{.5cm}	OCR1A=Atrigger; \\
	\textbraceright \\
	void TIMER1\_intcapture(uint16\_t capture) \\
	\textbraceleft \\
	\hspace*{.5cm}	ICR1=capture; \\
	\textbraceright \\
	/*****/
\end{minipage}
\vline \quad
\begin{minipage}[t]{.3\linewidth}
	\tiny
	void ADC\_ENABLE(uint8\_t ADC\_channel, uint8\_t ADC\_Vref, uint8\_t ADC\_lar, uint8\_t ADC\_Div) \\
	\textbraceleft \\
	\hspace*{.5cm}	uint8\_t ADMUX\_tmp; \\
	\hspace*{.5cm}	uint8\_t ADCSRA\_tmp; \\
	\hspace*{.5cm}	switch(ADC\_channel)\textbraceleft \\
	\hspace*{1cm}		case 0: \\
	\hspace*{1.5cm}			DDRC\&=\texttildelow (1$<<$PC0); \\
	\hspace*{1.5cm}			ADMUX\_tmp=0x00; \\
	\hspace*{1.5cm}			break; \\
	\hspace*{1cm}		case 1: \\
	\hspace*{1.5cm}			DDRC\&=\texttildelow (1$<<$PC1); \\
	\hspace*{1.5cm}			ADMUX\_tmp=0x01; \\
	\hspace*{1.5cm}			break; \\
	\hspace*{1cm}		case 2: \\
	\hspace*{1.5cm}			DDRC\&=\texttildelow (1$<<$PC2); \\
	\hspace*{1.5cm}			ADMUX\_tmp=0x02; \\
	\hspace*{1.5cm}			break; \\
	\hspace*{1cm}		case 3: \\
	\hspace*{1.5cm}			DDRC\&=\texttildelow (1$<<$PC3); \\
	\hspace*{1.5cm}			ADMUX\_tmp=0x03; \\
	\hspace*{1.5cm}			break; \\
	\hspace*{1cm}		case 4: \\
	\hspace*{1.5cm}			DDRC\&=\texttildelow (1$<<$PC4); \\
	\hspace*{.5cm}			ADMUX\_tmp=0x04; \\
	\hspace*{1.5cm}			break; \\
	\hspace*{1cm}		case 5: \\
	\hspace*{1.5cm}			DDRC\&=\texttildelow (1$<<$PC5); \\
	\hspace*{1.5cm}			ADMUX\_tmp=0x05; \\
	\hspace*{1.5cm}			break; \\
	\hspace*{1cm}		case 6: \\
	\hspace*{1.5cm}			ADMUX\_tmp=0x06; \\
	\hspace*{1.5cm}			break; \\
	\hspace*{1cm}		case 7: \\
	\hspace*{1.5cm}			ADMUX\_tmp=0x07; \\
	\hspace*{1.5cm}			break; \\
	\hspace*{1cm}		default: \\
	\hspace*{1.5cm}			ADMUX\_tmp=0x00; \\
	\hspace*{.5cm}	\textbraceright ; \\
	\hspace*{.5cm}	switch(ADC\_Vref)\textbraceleft \\
	\hspace*{1cm}		case 0: \\
	\hspace*{1.5cm}			break; \\
	\hspace*{1cm}		case 1: \\
	\hspace*{1.5cm}			ADMUX\_tmp \textbar =(1$<<$REFS0); \\
	\hspace*{1.5cm}			break; \\
	\hspace*{1cm}		case 3: \\
	\hspace*{1.5cm}			ADMUX\_tmp \textbar =(3$<<$REFS0); \\
	\hspace*{1.5cm}			break; \\
	\hspace*{1cm}		default: \\
	\hspace*{1.5cm}			ADMUX\_tmp \textbar =(1$<<$REFS0); \\
	\hspace*{.5cm}	\textbraceright ; \\
	\hspace*{.5cm}	switch(ADC\_lar)\textbraceleft \\
	\hspace*{1cm}		case 0: \\
	\hspace*{1.5cm}			break; \\
	\hspace*{1cm}		case 1: \\
	\hspace*{1.5cm}			ADMUX\_tmp \textbar =(1$<<$ADLAR); \\
	\hspace*{1.5cm}			break; \\
	\hspace*{1cm}		default: \\
	\hspace*{1.5cm}			break; \\
	\hspace*{.5cm}	\textbraceright ; \\
	\hspace*{.5cm}	switch(ADC\_Div)\textbraceleft \\
	\hspace*{1cm}		case 1: \\
	\hspace*{1.5cm}			ADCSRA\_tmp=(1$<<$ADPS0); \\
	\hspace*{1.5cm}			break; \\
	\hspace*{1cm}		case 2: \\
	\hspace*{1.5cm}			ADCSRA\_tmp=(1$<<$ADPS0); \\
	\hspace*{1.5cm}			break; \\
	\hspace*{1cm}		case 4: \\
	\hspace*{1.5cm}			ADCSRA\_tmp=(1$<<$ADPS1); \\
	\hspace*{1.5cm}			break; \\
	\hspace*{1cm}		case 8: \\
	\hspace*{1.5cm}			ADCSRA\_tmp=(3$<<$ADPS0); \\
	\hspace*{1.5cm}			break; \\
	\hspace*{1cm}		case 16: \\
	\hspace*{1.5cm}			ADCSRA\_tmp=(1$<<$ADPS2); \\
	\hspace*{1.5cm}			break; \\
	\hspace*{1cm}		case 32: \\
	\hspace*{1.5cm}			ADCSRA\_tmp=(5$<<$ADPS0); \\
	\hspace*{1.5cm}			break; \\
	\hspace*{1cm}		case 64: \\
	\hspace*{1.5cm}			ADCSRA\_tmp=(3$<<$ADPS1); \\
	\hspace*{1.5cm}			break; \\
	\hspace*{1cm}		case 128: \\
	\hspace*{1.5cm}			ADCSRA\_tmp=(7$<<$ADPS0); \\
	\hspace*{1.5cm}			break; \\
	\hspace*{1cm}		default: \\
	\hspace*{1.5cm}			ADCSRA\_tmp=(7$<<$ADPS0); \\
	\hspace*{1.5cm}			break; \\
	\hspace*{.5cm}	\textbraceright ; \\
	\hspace*{.5cm}	ADMUX=ADMUX\_tmp; \\
	\hspace*{.5cm}	ADCSRA\_tmp \textbar =(1$<<$ADIE); \\
	\hspace*{.5cm}	ADCSRA\_tmp \textbar =(1$<<$ADEN); \\
	\hspace*{.5cm}	ADCSRA\_tmp \textbar =(1$<<$ADSC); \\
	\hspace*{.5cm}	ADCSRA\_tmp \textbar =(1$<<$ADATE); \\
	\hspace*{.5cm}	ADCSRA=ADCSRA\_tmp; \\
	\hspace*{.5cm}	ADCSRB\&=\texttildelow (7$<<$ADTS0); \\
	\hspace*{.5cm}	sei(); \\
	\textbraceright \\
	/***Interrupt***/ \\
	ISR(TIMER0\_COMPA\_vect)\textbraceleft \\
	\hspace*{.5cm}	uint8\_t sREG=SREG; \\
	\hspace*{.5cm}	cli(); \\
	\hspace*{.5cm}	count- -; \\
	\hspace*{.5cm}	if(!count)\textbraceleft \\
	\hspace*{1cm}		PORTD \textasciicircum =(1$<<$PORTD6); \\
	\hspace*{1cm}		count=REPEAT; \\
	\hspace*{.5cm}	\textbraceright \\
	\hspace*{.5cm}	SREG=sREG; \\
	\hspace*{.5cm}	sei(); \\
	\textbraceright \\
	/******/ \\
	ISR(ADC\_vect) \\
	\textbraceleft \\
	\hspace*{.5cm}	adcL\_tmp=ADCL; \\
	\hspace*{.5cm}	adcH\_tmp=ADCH; \\
	\hspace*{.5cm}	adcL\_tmp=adcL\_tmp \textbar (adcH\_tmp$<<$8); \\
	\hspace*{.5cm}	TIMER1A\_trigger(adcH\_tmp); \\
	\textbraceright \\
	/***EOF***/
\end{minipage} \par
\begin{minipage}[T]{.45\linewidth}
	\normalsize
	\begin{tikzpicture}[node distance = 2cm, auto]
		% Define
		\tikzstyle{RECTANGLE_1} = [rectangle, align=left, rounded corners, minimum width=1cm, minimum height=1cm,text centered, draw=black, fill=green!30]
		\tikzstyle{RECTANGLE_2} = [rectangle, draw, align=left, fill=blue!10]
		\tikzstyle{RECTANGLE_3} = [rectangle, draw, align=left, fill=blue!40]
		\tikzstyle{RECTANGLE_4} = [rectangle, draw, align=left, fill=purple!60]
		\tikzstyle{RECTANGLE_5} = [rectangle, draw, fill=purple!60, text width=5em, text badly centered, minimum height=4em]
		\tikzstyle{RECTANGLE_6} = [rectangle, draw, align=left, fill=red!60]
		\tikzstyle{RECTANGLE_7} = [rectangle, draw, align=left, fill=orange!60]
		\tikzstyle{DIAMOND_1} = [diamond, draw, align=left, fill=red!80]
		\tikzstyle{ARROW} = [thick,->,>=stealth]
		\tikzstyle{LINE} = [draw, -latex']
		\tikzstyle{MYLINE} = [draw, ->,  thick, shorten <=4pt, shorten >=4pt]
		% Place node
		\node [RECTANGLE_1] (START) {MAIN};
		\node [RECTANGLE_2, below of=START] (SETUP) {PORTINIT();\\ TIMER0ASETUP();\\ count=REPEAT;\\ TIMER1A\_SETUP \\ ADC\_SETUP};
		\node [RECTANGLE_5, below of=SETUP] (LOOP) {While Loop};%
		% LINES and ARROWS
		\path [LINE] (START) -- node (l_1) {} (SETUP);
		\path [LINE] (SETUP) -- node (l_2) {} (LOOP);
		\path [LINE] (LOOP) |- ($(LOOP.south east) + (0.5,-0.5)$) |- (LOOP);
	\end{tikzpicture}
\end{minipage}
\quad
\begin{minipage}[T]{.45\linewidth}
	\normalsize
	\begin{tikzpicture}[node distance = 2cm, auto]
		% Define
		\tikzstyle{RECTANGLE_1} = [rectangle, align=left, rounded corners, minimum width=1cm, minimum height=1cm,text centered, draw=black, fill=green!30]
		\tikzstyle{RECTANGLE_2} = [rectangle, draw, align=left, fill=blue!10]
		\tikzstyle{RECTANGLE_3} = [rectangle, draw, align=left, fill=blue!40]
		\tikzstyle{RECTANGLE_4} = [rectangle, draw, align=left, fill=purple!60]
		\tikzstyle{RECTANGLE_5} = [rectangle, draw, fill=purple!60, text width=5em, text badly centered, minimum height=4em]
		\tikzstyle{RECTANGLE_6} = [rectangle, draw, align=left, fill=red!60]
		\tikzstyle{RECTANGLE_7} = [rectangle, draw, align=left, fill=orange!60]
		\tikzstyle{DIAMOND_1} = [diamond, draw, align=left, fill=red!80]
		\tikzstyle{ARROW} = [thick,->,>=stealth]
		\tikzstyle{LINE} = [draw, -latex']
		\tikzstyle{MYLINE} = [draw, ->,  thick, shorten <=4pt, shorten >=4pt]
		% Place node
		\node [RECTANGLE_1, below= 3cm of LOOP] (INTERRUPT_START) {TIMER0\_COMPA\_vect};
		\node [RECTANGLE_6, below of=INTERRUPT_START] (COUNT) {count- -};
		\node [DIAMOND_1, below of=COUNT] (COUNTNULL) {!count};
		\node [RECTANGLE_6, right=1cm of COUNTNULL] (RETURN) {RETURN};
		\node [RECTANGLE_7, left=1cm of COUNTNULL] (INIC_COUNTER) {PORTD\^{}=(1$<<$PORTD6) \\ count=REPEAT;};
		% LINES and ARROWS
		\path [LINE] (INTERRUPT_START) -- node (l_4) {} (COUNT);
		\path [LINE] (COUNT) -- node (l_5) {} (COUNTNULL);
		\path [LINE] (COUNTNULL) -- node (l_6) {no} (RETURN);
		\path [LINE] (COUNTNULL) -- node (l_7) {yes} (INIC_COUNTER);
		\path [LINE] (INIC_COUNTER) |- ($(INIC_COUNTER.south east) + (4,-1)$) -- (RETURN.south);
	\end{tikzpicture}
	\newline \newline
	\begin{tikzpicture}[node distance = 2cm, auto]
		% Define
		\tikzstyle{RECTANGLE_1} = [rectangle, align=left, rounded corners, minimum width=1cm, minimum height=1cm,text centered, draw=black, fill=green!30]
		\tikzstyle{RECTANGLE_2} = [rectangle, draw, align=left, fill=blue!10]
		\tikzstyle{RECTANGLE_3} = [rectangle, draw, align=left, fill=blue!40]
		\tikzstyle{RECTANGLE_4} = [rectangle, draw, align=left, fill=purple!60]
		\tikzstyle{RECTANGLE_5} = [rectangle, draw, fill=purple!60, text width=5em, text badly centered, minimum height=4em]
		\tikzstyle{RECTANGLE_6} = [rectangle, draw, align=left, fill=red!60]
		\tikzstyle{RECTANGLE_7} = [rectangle, draw, align=left, fill=orange!60]
		\tikzstyle{DIAMOND_1} = [diamond, draw, align=left, fill=red!80]
		\tikzstyle{ARROW} = [thick,->,>=stealth]
		\tikzstyle{LINE} = [draw, -latex']
		\tikzstyle{MYLINE} = [draw, ->,  thick, shorten <=4pt, shorten >=4pt]
		% Place node
		\node [RECTANGLE_1, below= 3cm of MAIN] (INTERRUPT_START) {ADC\_vect};
		\node [RECTANGLE_6, below of=INTERRUPT_START] (DEC18) {OCR1AL=ADCH};
		\node [RECTANGLE_6, below of=DEC18] (RETI) {RETURN};
		% LINES and ARROWS
		\path [LINE] (INTERRUPT_START) -- node (l_4) {} (DEC18);
		\path [LINE] (DEC18) -- node (l_5) {} (RETI);
	\end{tikzpicture}
\end{minipage}
%\newpage
\\
\\
\\
\\
Neste caso exagerei um bocado pois fiz umas funções genérica para preencher os respetivos registos de configuração.
\section{Resultados}
No geral foi conseguido os objetivos do relat\'{o}rio, o manuseamento e parametriza\c{c}\~{a}o do Atmega88 com o auxilio do seu datasheet, e a ajuda das ferramentas Atmel ICE e o Atmel Studio 7.
Foi de forma a ter em conta para o c\'{o}digo ser percet\'{i}vel e ao mesmo tempo amigo do utilizador, criado uma abstra\c{c}\~{a}o simples e l\'{o}gica na perspetiva humana, de fácil acompanhamento. \\
Na parte de hardware a pouco a dizer devido aos circuitos utilizados j\'{a} serem muito conhecidos na electr\'{o}nica, e se surgir duvidas facilmente com pesquisa h\'{a} centenas de fontes de inform\c{c}\~{a}o ao dispor como a Internet ou literatura.
%\newpage
\section{Conclus\~{o}es}
O assembly \'{e} maquina dependente e de programa\c{c}\~{a}o detalhada muito ligada ao hardware, sua programa\c{c}\~{a}o em esparguete que n\~{a}o \'{e} aconselh\'{a}vel na linguagem C como por exemplo a instru\c{c}\~{a}o {\bf goto} em {\bf C}, da origem a c\'{o}digo dif\'{i}cil de seguir e se perceber.
O {\bf C} \'{e} mais liberal maquina independente, mas o utilizador n\~{a}o tem controle sobre o mapeamento das vari\'{a}veis na memoria nem na sua compila\c{c}\~{a}o. A linguagem tamb\'{e}m \'{e} consebido com uma estrutura mais l\'{o}gica sendo sua transfer\^{e}ncia para {\bf Flowchart} intuitiva. \par
Fazendo {\bf Delays} ou {\bf Polling} interrompe o programa ficando a espera, dai as interrup\c{c}\~{o}es s\~{a}o mais \'{u}teis, s\'{o} chamando sua rotina quando necess\'{a}rio. \\
	
O dom\'{i}nio de interpreta\c{c}\~{a}o e manipula\c{c}\~{a}o do datasheet em conjunto com seu component cosidero como objetivo priorit\'{a}rio, perceber seus conceitos e modos de funcionamento, interpretar os esquemas electr\'{o}nicos e sua interliga\c{c}\~{a}o com a parte de programa\c{c}\~{a}o, suas defini\c{c}\~{o}es e caracter\'{i}sticas ter uma vis\~{a}o do que \'{e} poss\'{i}vel fazer da combina\c{c}\~{a}o de hardware e software, tendo em conta suas limita\c{c}\~{o}es e virtudes, podemos com esta aprendizagem facilmente nos adaptar a outras arquiteturas e fabricantes, o hardware em principio ser\'{a} igual o modo de o manipular por software o paradigma de aproxima\c{c}\~{a}o poder\'{a} ser diferente mas a eletr\'{o}nica em principio \'{e} universal.\\ \\
Este trabalho esta publicado no github, link abaixo: \\
{\bf \textit https://github.com/sergio1020881/LABSIS20202021}
%%%%%%%%%%%%%%%%%%%%%%%%%%%%%%%%%%%%%%%%%%%%%%%%%%%%%%%%%%%%%%%%%%%%%%%%%%%%%%%%%%%%%%%%%%%%%%%%%%%
%%%%%%%%%%%%%%%%%%%%%%%%%%%%%%%%%%%%%%%%%%%%%%%%%%%%%%%%%%%%%%%%%%%%%%%%%%%%%%%%%%%%%%%%%%%%%%%%%%%
%%%%%%%%%%%%%%%%%%%%%%%%%%%%%%%%%%%%%%%%%%%%%%%%%%%%%%%%%%%%%%%%%%%%%%%%%%%%%%%%%%%%%%%%%%%%%%%%%%%
%%%%%%%%%%%%%%%%%%%%%%%%%%%%%%%%%%%%%%%%%%%%%%%%%%%%%%%%%%%%%%%%%%%%%%%%%%%%%%%%%%%%%%%%%%%%%%%%%%%
%%%%%%%%%%%%%%%%%%%%%%%%%%%%%%%%%%%%%%%%%%%%%%%%%%%%%%%%%%%%%%%%%%%%%%%%%%%%%%%%%%%%%%%%%%%%%%%%%%%
%%%%%%%%%%%%%%%%%%%%%%%%%%%%%%%%%%%%%%%%%%%%%%%%%%%%%%%%%%%%%%%%%%%%%%%%%%%%%%%%%%%%%%%%%%%%%%%%%%%
\newpage
\input{./input/DEFINICAO}
%Figuras Bibliografia Index
%\newpage
%\listoffigures
%
\cite{*}
\bibliography{./bibliography/Bibliography}
%outro metodo mas manual\input{Bibliografia}
%
%\printindex
%
\newpage
\footnote{Apontamentos}
%
	\end{document}
%%%EOF%%%
