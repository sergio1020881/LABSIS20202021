\input{./input/PREAMBLEREPORT}
%%%%%%%%%%%%%%%%%%%%%%%%%%%%%%%%%%%%%%%%%%%%%%%%%%%%%%%%%%%%%%%%%%%%%%%%%%%%%%%
\begin{document}
%\bibliographystyle{apacite}
\bibliographystyle{babplain}
%\bibliographystyle{bstfilename}
%%%%%%%%%%%%%%%%%%%%%FIX SECTION NUMBERING IN CASE REPORT%%%%%%%%%%%%%%%%%%%%%%
\renewcommand\thesection{\arabic{section}}
\renewcommand\thesubsection{\thesection.\arabic{subsection}}
\renewcommand\thesubsubsection{\thesection.\thesubsection.\arabic{subsubsection}}
%%%%%%%%%%%%%%%%%%%%%%%%%%%%%%%%%%%%%%%%%%%%%%%%%%%%%%%%%%%%%%%%%%%%%%%%%%%%%%%%
\begin{titlepage}
\title{Posicionamento Painel Fotovoltaico}
\author{
\\
\\
\\
\\
\\
\\
\begin{minipage}{0.4\linewidth}
\flushleft
\textbf{Aluno} : \\
\emph{S\'{e}rgio Santos},\;$N^o$:\; 1020881 \\
\end{minipage}
\hfill
\fbox{
\begin{minipage}{0.5\linewidth}
\centering
\textbf{Docente/Orientador} \\
\begin{flushleft}
Abel António de Azevedo Ferreira, \textit{abe}\\
Lino Manuel Baptista Figueiredo, \textit{lbf}\\
\end{flushleft}
\textbf{Unidade Curricular} \\
LABSIS \\
\end{minipage}
}
}
\date{\rule[180pt]{0pt}{5pt} } %\today
\end{titlepage}
%%%%%%%%%%%%%%%%%%%%%%%%%%%%%%%%%%%%%%%%%%%%%%%%%%%%%%%%%%%%%%%%%%%%
\begin{minipage}{\linewidth}
\includegraphics[scale=0.60]{./image/capa/ISEP_marca_cor_grande.png}
\maketitle
\end{minipage}

\tableofcontents
\appendix
\pagestyle{plain} %plain headings empty
%\setcounter{chapter}{0}
%\numberwithin{page}{section}
%\renewcommand{\abstractname}{Executive Summary}
\setlength{\parindent}{0in}
%%%%%%%%%%%%%%%%%%%%%%%%%%%%%%%%%%%%%%%%%%%%%%%%%%%%%%%%%%%%%%%%%%%%%%%%%%%%%%%%
\label{Resumo}
\begin{abstract}
\qquad O projeto que se pretende fazer é um sistema de posicionamento de painel fotovoltaico para obter maior rendimento de produção de energia, já que é conhecido que se pode tirar proveito até 40\% de energia que é dependente na estação do ano. \\
\\
O objetivo é desenvolver um meio de controlo tendo em mente ser um sistema Stand Alone de tamanho considerável, não vai ser concentrado ao redor do painel fotovoltaico em si e suas características ou modos de funcionamento, mas apenas o sistema de Sun Track, também supondo que tem baterias de carga com o intuito que o sistema possa ser autónomo. \\
\\
O projeto é apenas académico e de simulação em escala pequena ou de bancada.
\\
\vspace{15cm}\\
\textbf{Palavras Chave:} Sensores, Componentes, Motores
\end{abstract}
%%%%%%%%%%%%%%%%%%%%%%%%%%%%%%%%%%%%%%%%%%%%%%%%%%%%%%%%%%%%%%%%%%%%%%%%%%%%%%%%
\newpage
\section{Introdução}
%%%%%%%%%%%%%%%%%%%%%%%%%%%%%%%%%%%%%%%%%%%%%%%%%%%%%%%%%%%%%%%%%%%%%%%%%%%%%%%%


\begin{figure}[H]
\centering
\includegraphics[scale=0.7]{./image/Mount_Method.jpg}
\caption{Tipos de Montagem \cite{book_2}}
\end{figure}

\newpage



More energy is collected by the end of the day if the PV module is installed on a tracker with an actuator that follows the sun. There are two types of sun trackers:\\
• One-axis tracker, which follows the sun from east to west during the day.\\
• Two-axis tracker, which follows the sun from east to west during the day, and from north to south during the seasons of the year (Figure 9.18).\cite{book_2} \\

A sun-tracking design can increase the energy yield up to 40\% over the year compared to the fixed-array design. Dual-axis tracking is done by two linear actuator motors, which follow the sun within one degree of accuracy (Figure 9.19). During the day, it tracks the sun east to west. At night it turns east to position itself for the next morning’s sun. Old trackers did this after sunset using a small nickel-cadmium battery. The new design eliminates the battery requirement by doing the turning in the weak light of the dusk and/or dawn. The Kelly cosine presented in Table 9.1 is useful in assessing accurately the power available in sunlight incident at extreme angles in the morning or evening. When a dark cloud obscures the sun, the tracker may aim at the next brightest object, which is generally the edge of a cloud. When the cloud is gone, the tracker aims at the sun once again, and so on and so forth. Such sun hunting is eliminated in newer sun trackers. \cite{book_2}\\

\newpage

\begin{figure}[H]
\centering
\includegraphics[scale=0.52]{./image/Sensor_Actuator_Principle.jpg}
\caption{Principio de Funcionamento Sensor \cite{book_2}}
\end{figure}

One method of designing the sun tracker is to use two PV cells mounted on two 45° wedges , and connecting them differentially in series through an actuator motor. When the sun is normal, the currents on both cells are equal to $I_o \cos (45^o)$. \\

As they are connected in series opposition, the net current in the motor is zero, and the array stays put. On the other hand, if the array is not normal to the sun, the sun angles on the two cells are different, giving two different currents as follows: \\
$I_1 = I_o \cos(45 + \delta) and I_2 = I_o \cos(45 - \delta)$ \\
The motor current is therefore: \\
$I_m = I_1 - I_2 = I_o \cos(45 + \delta) - I_o \cos(45 - \delta)$ \\
We can express the two currents as follows: \\
$I_1 = I_o \cos (45) - I_o \delta \sin (45) and I_2 = I_o \cos (45) + I_o \delta sin (45)$ \\
The motor current is then\\
\\
$I_m = I_1 - I_2 = 2 I_o \quad \delta \sin (45) = \sqrt{2} I_o \delta \quad if \quad \delta \quad is \quad in \quad radians$ \\
A small pole-mounted panel can use one single-axis or dual-axis sun tracker. A large array, on the other hand, is divided into small modules, each mounted on its own sun tracker. This simplifies the structure and eliminates the problems related to a large movement in a large panel.



\newpage

The battery stores energy in an electrochemical form and is the most widely used device for energy storage in a variety of applications. The electrochemical energy is in a semiordered form, which is in between the electrical and thermal forms. It has a one-way conversion efficiency of 85 to 90\%.
There are two basic types of electrochemical batteries:
The primary battery, which converts chemical energy into electric energy.\\
The electrochemical reaction in a primary battery is nonreversible, and the battery is discarded after a full discharge. For this reason, it finds applica- tions where a high energy density for one-time use is required.
The secondary battery, which is also known as the rechargeable battery.
The electrochemical reaction in the secondary battery is reversible. After a discharge, it can be recharged by injecting a direct current from an external source. This type of battery converts chemical energy into electric energyin the discharge mode. In the charge mode, it converts the electric energy into chemical energy. In both modes, a small fraction of energy is converted into heat, which is dissipated to the surrounding medium. The round-trip conversion efficiency is between 70 and 80\%.

The internal construction of a typical electrochemical cell is shown in Figure 10.1. It has positive and negative electrode plates with insulating separators and a chemical electrolyte in between. The two groups of electrode plates are connected to two external terminals mounted on the casing. The cell stores electrochemical energy at a low electrical potential, typically a few volts. The cell capacity, denoted by C , is measured in ampere-hours (Ah), meaning it can deliver C A for one hour or C / n A for n hours.
The battery is made of numerous electrochemical cells connected in a series–par-allel combination to obtain the desired battery voltage and current. The higher the battery voltage, the higher the number of cells required in series. The battery rating is stated in terms of the average voltage during discharge and the ampere-hour capacity it can deliver before the voltage drops below the specified limit. The product of the voltage and ampere-hour forms the watthour (Wh) energy rating the battery
can deliver to a load from the fully charged condition. The battery charge and discharge rates are stated in units of its capacity in Ah. For example, charging a 100-Ah battery at C
/10 rate means charging at 100/10 = 10 A. Discharging that
battery at C /2 rate means drawing 100/2 = 50 A, at which rate the battery will be fully discharged in 2 h. The state of charge ( SOC ) of the battery at any time is defined as the following:
FIGURE 10.1
Electrochemical energy storage cell construction.
$SOC = \frac{Ah capacity remaning in the battery}
{Rated Ah capacity}$





\newpage




\newpage

\begin{minipage}[t]{.60\linewidth}
%	\quad List 1:
	\begin{itemize}
		\setlength\itemsep{-0.3em}
		\item ET-BASE AVR ATmega128 \\
		Mainboard com MCU atmega 128 da ETT. \\
		\item LDR - Light Controlled Resistor \\
		Dark resistance 1M.\\
		40k at 10 lux.\\
		Max voltage 100V.\\
		Max power 80mW.\\
		\item INA128PA - Instrumentation Amplifier \\
		Banda de transmissão: 1.3MHz \\
		Montagem: THT \\
		Número de canais: 1 \\
		Carcaça: DIP8 \\
		Rapidez de subida de tensão: 4V / $\mu$ s \\
		Temperatura de trabalho: -40...85°C \\
		Entradas de tensão instável: 0.025mV \\
		Tensão de trabalho: 2.25...18V \\
		\item LCD 20x4 Blue NHD-0420DZ-NSW-BBW \\
		4x20 Characters HD44780 compatible\\
	\end{itemize}
\end{minipage}
\begin{minipage}[t]{.31\linewidth}
%	\quad List 2:
	\begin{itemize}
		\setlength\itemsep{-0.3em}
		\item Model Craft RS-2 \- Servo Motor \\
		Control System:\\
		+Pulse Width Control 1500 $\mu$ sec \\
		Operating Voltage:\\ 
		4.8-6.0 Volts \\
		Operating Speed (4.8V):\\
		0.19sec/60° at no load \\
		Stall Torque (4.8V):\\
		42 oz/in (3.0 kg/cm) \\
		\item PCF8563 RTC Board \\
		Real-time clock/calendar function. \\ 
		Battery on board. \\
		I2C communication. \\
		\item KEYPAD4X4W \\
		16 Button Keypad switch. \\
	\end{itemize}
\end{minipage}\\
{\it https://www.ptrobotics.com/} \\
{\it https://www.futurlec.com/index.shtml} \\








%\begin{figure}[H]
%\centering
%\includegraphics[scale=0.52]{./image/OB/OB_contributions.jpg}
%\caption{Contribuições para OB \cite{book_2}}
%\end{figure}
%\cite{book_10}\\
%\begin{figure}[H]
%\begin{minipage}{0.3\linewidth}
%\flushleft
%\includegraphics[scale=0.30]{./image/OB/OB_MUltilevelmodelCulture.jpg}
%\end{minipage}
%\hspace{1cm}
%\begin{minipage}{0.4\linewidth}
%\flushleft
%\includegraphics[scale=0.44]{./image/OB/Hofstede_pt}
%\end{minipage}
%\caption{Modelo de Multi-níveis \cite{book_11} da cultura e Modelo Hofstede, Portugal}
%\end{figure}
%%%%%%%%%%%%%%%%%%%%%%%%%%%%%
%\textcolor{blue}{Visão} e seus \textcolor{blue}{Valores}.\\
%%%%%%%%%%%%%%%%%%%%%%%%%%%%%
%\textcolor{blue}{missão} 
%\begin{figure}[H]
%\flushleft
%\captionsetup{justification=raggedright,singlelinecheck=false}
%\includegraphics[scale=0.4]{./image/OB/OC_Dimensions.jpg}
%\caption{Dimensões da Cultura Organizacional \cite{book_4}}
%\end{figure}\par
\newpage
%%%%%%%%%%%%%%%%%%%%%%%%%%%%%%%%%%%%%%%%%%%%%%%%%%%%%%%%%%%%%%%%%%%%%%%%%%%%
\section{Organização}
%\begin{figure}[ht]
%\begin{center}
%\includegraphics[scale=0.5]{./image/ROQ/maquinas/ECO-P18_600x600-2-275x275.jpg}
%\includegraphics[scale=0.5]{./image/ROQ/maquinas/EVO-600x600-275x275.jpg}
%\includegraphics[scale=0.5]{./image/ROQ/maquinas/nanop10-275x275.jpg}
%\includegraphics[scale=0.5]{./image/ROQ/maquinas/NEXTP18-600x6001-275x275.jpg}
%\includegraphics[scale=0.5]{./image/ROQ/maquinas/PRO-600x600-275x275.jpg}
%\includegraphics[scale=0.5]{./image/ROQ/maquinas/You-600x600-275x275}
%\end{center}
%\caption{Produtos Principais}
%\end{figure}
%%%%%%%%%%%%%%%%%%%%%%%%%%%%%%%%%%%%%%%%%%%%%%%%%%%%%%%%%%%%%%%%%%%%%%%%%%%%
\newpage
\section{Equipamento}
%%%%%%%%%%%%%%%%%%%%%%%%%%%%%%%%%%%%%%%%%%%%%%%%%%%%%%%%%%%%%%%%%%%%%%%%%%%%
%\vspace{1cm}\\
%\begin{figure}[H]
%\centering
%\includegraphics[scale=.5]{./image/OB/Leadership.jpg}\\
%\caption{Inquérito do Estilos de Liderança \cite{article_1}}
%\end{figure}\par
%%%%%%%%%%%%%%%%%%%%%%%%%%%%%%%%%%%%%%%%%%%%%%%%%%%%%%%%%%%%%%%%%%%%%%%%%%%%%%%%%%%%%%%%%%%%%%%%%%%%%%%%%%%%%%%
\newpage
%\vspace{1cm}
%\begin{figure}[H]
%\centering
%\includegraphics[scale=.6]{./image/OB/Culture.jpg}\\
%\caption{Inquérito do tipo de Cultura Organizacional \cite{article_1}}
%\end{figure}\par
%\begin{figure}[H]
%\centering
%\includegraphics[scale=.35]{./image/OB/Ogbonna_Harris.jpg}\\
%\caption{Modelo Ogbonna \& Harris \cite{article_1}}
%\label{Modelo}
%\end{figure}\par
%%%%%%%%%%%%%%%%%%%%%%%%%%%%%%%%%%%%%%%%%%%%%%%%%%%%%%%%%%%%%%%%%
\newpage
\section{Conclusões}

%%%%%%%%%%%%%%%%%%%%%%%%%%%%%%%%%%%%%%%%%%%%%%%%%%%%%%%%%%%%%%%%%%
\newpage
%%%%%%%%%%%%%%%%%%%%%%%%%%%%%%%%%%%%%%%%%%%%%%%%%%%%%%%%%%%%%%%%%%
%Figuras Bibliografia Index
\listoffigures
\cite{*}
\bibliography{./bibliography/Bibliography}
%\printindex
\newpage
\footnote{Apontamento}
\end{document}
%%%%%%%%%%%%%%%%%%%%%%%%%%%%%%%%%%%%%%%%%%%%%%%%%%%%%%%%%%%%%%%%%%
\begin{comment}
\textcolor{orange}{O}pportunity (oportunidades)
\textcolor{purple}{I}nterno
\end{comment}